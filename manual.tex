\documentclass{article}
\usepackage[utf8]{inputenc} %codification of the document
\usepackage{hyperref} % reference-package
\usepackage[table]{xcolor}% http://ctan.org/pkg/xcolor
\usepackage[a4paper, total={6in, 8in}]{geometry}

%Here begins the body of the document
\begin{document}

\title{TBarrier notebooks}

\author{George Haller and Alex P. Encinas Bartos}

\maketitle

\tableofcontents

\section{Introduction}
\textit{TBarrier} contains a series of jupyter notebooks that guide you  through methods used to extract advective, diffusive, stochastic and active  transport barriers from discrete velocity data. \\ \\
The collection of notebooks implement the methods explained in the book \href{https://github.com/EncinasBartos/TBarrier}{Transport Barriers and Coherent Structures in Flow Data} written by George Haller. We address both two-dimensional and three dimensional steady/unsteady velocity fields. Hence, \textit{TBarrier} is designed for fluid dynamicists in general, with a special focus on oceanographers and CFD analysts.

\subsection{Programming Language}
The jupyter notebooks are written in \href{https://www.python.org}{Python}. It was, however, designed in a way so that it could easily be translated into other languages such as \href{https://julialang.org}{Julia} and \href{https://www.mathworks.com/products/matlab.html}{Matlab}. Table \ref{tab: language} includes a qualitative (and hence subjective) comparison of Python, Julia and Matlab.
\\ \\ \\
\begin{table}
\centering
\begin{tabular}{ |p{3cm}||p{2cm}|p{2cm}|p{2cm}|  }
 \hline
 \multicolumn{4}{|c|}{Qualitative Comparison of programming languages} \\
 \hline
Feature & Python  & Julia & Matlab \\
 \hline
Access   &  \cellcolor{green!25} Open Source    & \cellcolor{green!25} Open Source &   \cellcolor{red!25} Closed Source\\
 \hline
Support from Community &  \cellcolor{green!25} Good  & \cellcolor{red!25} Bad &   \cellcolor{yellow!25} Moderate \\
 \hline
ODE solver &  \cellcolor{red!25} Bad  & \cellcolor{green!25} Good &   \cellcolor{yellow!25} Moderate \\
 \hline
 Computational speed &  \cellcolor{yellow!25} Moderate  & \cellcolor{green!25} Good &   \cellcolor{yellow!25} Moderate \\
 \hline
  User friendly &  \cellcolor{green!25} Good  & \cellcolor{red!25} Bad &   \cellcolor{yellow!25} Moderate \\
 \hline
\end{tabular}
\caption{Qualitative Comparison of programming languages}
\label{tab: language}
\end{table}
The decision on the programming language was based on various factors. Although computational speed is definitely a criterion that has to be considered when implementing the code, other factors such as the ease of access, user friendliness and versatility of the programming language have to be included. Although Python seems to be behind Julia  and Matlab in terms of computational speed (especially when solving a large number of ODEs), it outperforms them when it comes to user friendliness. Apart from academia, Python is also widely used in the industry as it is Open Source. Due to the limitations on the ODE solver, the computational speed of the computations in the \textit{TBarrier} notebook is lower compared to Matlab or Julia. However, the main purpose of the notebook series is to explain the methods highlighted in the book with clarity on a very user friendly and easily accessible platform.

\subsection{Existing Software}
We highlight three major publicly available existing software packages designed to extract transport barriers from discrete velocity data. \\ \\
The toolbox \textit{\href{https://coherentstructures.github.io/CoherentStructures.jl/stable/}{CoherentStructures.jl}} has been developed in Oliver Junge's research group at TUM, Germany. It is written in Julia and computationally very efficient. It focuses, however, principally on advective transport barriers in two dimensional flows and does not focus on methods for computing diffusive, stochastic and active barriers. \\ 

Secondly, we point out the \textit{\href{https://github.com/haller-group/LCStool}{LCStool}} and the \textit{\href{https://github.com/haller-group/BarrierTool}{BarrierTool}} developed by \href{http://georgehaller.com/}{George Haller's research group} at ETHZ. Both tools are written in Matlab. The former tool focuses on the extraction of advective Lagrangian Coherent Structures from two-dimensional discrete velocity data, whereas the latter includes a GUI addressing both advective and diffusive Lagrangian and Eulerian transport barriers in two-dimensional flows.

\section{Overview \textit{TBarrier} notebooks}

We now outline the structure of the \href{https://github.com/EncinasBartos/TBarrier}{\textit{\underline{TBarrier}}} repository.

\begin{enumerate}
\item  \href{https://github.com/EncinasBartos/TBarrier/tree/main/TBarrier/2D}{\underline{2D}}
\begin{enumerate}
\item \href{https://github.com/EncinasBartos/TBarrier/tree/main/TBarrier/2D/data}{\underline{data}}
\begin{enumerate}
\item \href{https://github.com/EncinasBartos/TBarrier/tree/main/TBarrier/2D/data/AVISO}{\underline{AVISO}}: \\ \\
Tutorial on how to download and preprocess AVISO data. The data is then stored in a 'mat' file. \\
\item \href{https://github.com/EncinasBartos/TBarrier/tree/main/TBarrier/2D/data/BickleyJet}{\underline{BickleyJet}}: \\ \\
Bickley jet velocity data is discretized from its analytic formulation. The data is then stored in a 'mat' file. \\ \\
\item \href{https://github.com/EncinasBartos/TBarrier/tree/main/TBarrier/2D/data/Turbulence}{\underline{Isotropic two-dimensional turbulence}} \\ \\
Isotropic two-dimensional turbulence data is stored in a 'mat' file. \\ \\
\end{enumerate}
\item \href{https://github.com/EncinasBartos/TBarrier/tree/main/TBarrier/2D/demos}{\underline{demos}} \\
\begin{enumerate}
\item \href{https://github.com/EncinasBartos/TBarrier/tree/main/TBarrier/2D/demos/AdvectiveBarriers}{\underline{AdvectiveBarriers}} \\
\begin{itemize}
\item \href{https://github.com/EncinasBartos/TBarrier/tree/main/TBarrier/2D/demos/AdvectiveBarriers/FTLE}{\underline{FTLE}}: (see $ \textit{Hyperbolic LCS from the finite-time Lyapunov exponent} $) \\ \\
FTLE-field applied to AVISO, Bickley and 2D turbulence data. \\ \\
\item \href{https://github.com/EncinasBartos/TBarrier/tree/main/TBarrier/2D/demos/AdvectiveBarriers/PRA}{\underline{PRA}}: (see $ \textit{Elliptic LCSs from the polar rotation angle} $)\\ \\ PRA-field applied to AVISO, Bickley and 2D turbulence data. \\ \\
\item \href{https://github.com/EncinasBartos/TBarrier/tree/main/TBarrier/2D/demos/AdvectiveBarriers/LAVD}{\underline{LAVD}}: (see $ \textit{Elliptic LCSs from the Lagrangian-averaged vorticity deviation} $ and $ \textit{ LAVD for 2D flows} $)\\ \\
LAVD-field applied to AVISO, Bickley and 2D turbulence data. \\ \\
\item \href{https://github.com/EncinasBartos/TBarrier/tree/main/TBarrier/2D/demos/AdvectiveBarriers/TRA}{\underline{TRA}}: (see $ \textit{Quasi-objective, single-trajectory diagnostics for transport
barriers} $) \\ \\
TRA-field applied to AVISO, Bickley and 2D turbulence data. \\ \\
\item \href{https://github.com/EncinasBartos/TBarrier/tree/main/TBarrier/2D/demos/AdvectiveBarriers/TSE}{\underline{TSE}}: (see $ \textit{Quasi-objective, single-trajectory diagnostics for transport
barriers} $) \\ \\
TSE-field applied to AVISO, Bickley and 2D turbulence data. \\ \\
\item \href{https://github.com/EncinasBartos/TBarrier/tree/main/TBarrier/2D/demos/AdvectiveBarriers/HyperbolicLCS}{\underline{HyperbolicLCS}}: (see $ \textit{Local variational theory of hyperbolic LCS} $ $ \textit{Hyperbolic LCSs in 2D flows} $)\\ \\
Hyperbolic LCS from tensorlines (shrinklines/stretchlines) of the Cauchy-Green strain tensor applied to AVISO, Bickley and 2D turbulence data. \\ \\
\item \href{https://github.com/EncinasBartos/TBarrier/tree/main/TBarrier/2D/demos/AdvectiveBarriers/HyperbolicOECS}{\underline{HyperbolicOECS}}: (see $ \textit{Shearless OECSs and objective saddle points in 2D flows} $)\\ \\
Hyperbolic OECS from (local) tensorlines launched from objective saddle points applied to AVISO. \\ \\
\item \href{https://github.com/EncinasBartos/TBarrier/tree/main/TBarrier/2D/demos/AdvectiveBarriers/EllipticLCS}{\underline{EllipticLCS}}: (see $ \textit{Computing elliptic LCSs as closed null-geodesics} $)\\ \\
Elliptic LCS as closed null geodesics from the Cauchy-Green strain tensor applied to AVISO, Bickley and 2D turbulence data. \\ \\
\item \href{https://github.com/EncinasBartos/TBarrier/tree/main/TBarrier/2D/demos/AdvectiveBarriers/EllipticOECS}{\underline{EllipticOECS}}: ($ \textit{Computing elliptic LCSs as closed null-geodesics} $) \\ \\
Elliptic OECS as closed null geodesics from the rate of strain tensor applied to AVISO.
\\ \\
\item \href{https://github.com/EncinasBartos/TBarrier/tree/main/TBarrier/2D/demos/AdvectiveBarriers/FastTensorlineComputation}{\underline{FastTensorlineComputation}}: \\ \\

Computation of tensorlines using the newly proposed algorithm with the re-parametrization of the eigenvectors. The Fast Tensorline Computation (FTC) is applied to the rate of strain tensor to (locally) extract hyperbolic OECS away from tensorline singularities.\\ \\

\item \href{https://github.com/EncinasBartos/TBarrier/tree/main/TBarrier/2D/demos/AdvectiveBarriers/PoincareMap}{\underline{PoincareMap}}: (see $ \textit{Poincaré maps} $ and $ \textit{Elliptic LCS in 2D: Black-hole vortices} $)\\ \\
Poincare map is used to locate elliptic LCS/OECS from closed shearlines in AVISO data. \\ \\
\end{itemize}
\item \href{https://github.com/EncinasBartos/TBarrier/tree/main/TBarrier/2D/demos/DiffusionBarriers}{\underline{DiffusionBarriers}}: \\ \\
In the following subfolder we discuss diffusive Lagrangian and Eulerian transport barriers. The analysis and algorithm are similar to the ones discussed for the advective Barriers. \\ \\
\begin{itemize}
\item \href{https://github.com/EncinasBartos/TBarrier/tree/main/TBarrier/2D/demos/DiffusionBarriers/DBS}{\underline{DBS}}: (see $ \textit{ Unconstrained diffusion barriers in 2D flows} $) \\ \\
Diffusion Barrier Sensitivity (DBS) applied to AVISO and Bickley data \\ \\
\item \href{https://github.com/EncinasBartos/TBarrier/tree/main/TBarrier/2D/demos/DiffusionBarriers/EllipticLagrangianDiffusionBarriers}{\underline{EllipticLagrangianDiffusionBarriers}}: (see $ \textit{ Unconstrained diffusion barriers in 2D flows} $) \\ \\
Elliptic Lagrangian Diffusion Barriers extracted from AVISO and Bickley data. \\ \\
\item \href{https://github.com/EncinasBartos/TBarrier/tree/main/TBarrier/2D/demos/DiffusionBarriers/EllipticEulerianDiffusionBarriers}{\underline{EllipticEulerianDiffusionBarriers}}: (see $ \textit{ Unconstrained diffusion barriers in 2D flows} $) \\ \\
Elliptic Eulerian Diffusion Barriers extracted from AVISO and Bickley data. \\ \\
\item \href{https://github.com/EncinasBartos/TBarrier/tree/main/TBarrier/2D/demos/DiffusionBarriers/Tensorlines}{\underline{Tensorlines}}: \\ \\
Tensorlines computed from the averaged diffusive Cauchy Green strain tensor extracted from AVISO and Bickley data.\\ \\
\end{itemize}
\item \href{https://github.com/EncinasBartos/TBarrier/tree/main/TBarrier/2D/demos/StochasticBarriers}{\underline{StochasticBarriers}}: (see $ \textit{Transport barriers in stochastic velocity fields} $)\\ \\
Barriers in stochastic velocity field extracted from AVISO data. We specifically focus on elliptic Lagrangian stochastic Barriers \\ \\
\item \href{https://github.com/EncinasBartos/TBarrier/tree/main/TBarrier/2D/demos/ActiveBarriers}{\underline{ActiveBarriers}}: (see $ \textit{2D homogeneous, isotropic turbulence} $)\\ \\
The active barriers are extracted from the two-dimensional turbulence simulation. Lagrangian end Eulerian active barriers to vorticity and linear momentum are extracted using the active FTLE, active PRA, active TSE, active TRA and the Hamiltonian based formulation. \\
\begin{itemize}
\item \href{https://github.com/EncinasBartos/TBarrier/tree/main/TBarrier/2D/demos/ActiveBarriers/aFTLE}{\underline{aFTLE}}: (see $ \textit{Active FTLE (aFTLE) and active TSE (aTSE)} $) \\
\item \href{https://github.com/EncinasBartos/TBarrier/tree/main/TBarrier/2D/demos/ActiveBarriers/aTSE}{\underline{aTSE}}: (see $ \textit{Active FTLE (aFTLE) and active TSE (aTSE)} $) \\
\item \href{https://github.com/EncinasBartos/TBarrier/tree/main/TBarrier/2D/demos/ActiveBarriers/aTRA}{\underline{aTRA}}: (see $ \textit{Active PRA (aPRA) and active TRA (aTRA)} $) \\
\item \href{https://github.com/EncinasBartos/TBarrier/tree/main/TBarrier/2D/demos/ActiveBarriers/aPRA}{\underline{aPRA}}: (see $ \textit{Active PRA (aPRA) and active TRA (aTRA)} $) \\
\item \href{https://github.com/EncinasBartos/TBarrier/tree/main/TBarrier/2D/demos/ActiveBarriers/Hamiltonian}{\underline{Hamiltonian}}: (see $ \textit{Active transport barriers in general 2D Navier–Stokes flow} $)\\
\end{itemize}
\end{enumerate}
\item \href{https://github.com/EncinasBartos/TBarrier/tree/main/TBarrier/2D/subfunctions}{\underline{subfunctions}}: \\ \\
This folder contains frequently used functions to compute trajectories from two-dimensional velocity data, evaluate the gradient of the flowmap/velocity, the classic Cauchy-Green strain tensor, etc... \\ \\
\end{enumerate}
\item  \href{https://github.com/EncinasBartos/TBarrier/tree/main/TBarrier/3D}{\underline{3D}}
\begin{enumerate}
 \item \href{https://github.com/EncinasBartos/TBarrier/tree/main/TBarrier/3D/data}{\underline{data}}
 \begin{enumerate}
 \item \href{https://github.com/EncinasBartos/TBarrier/tree/main/TBarrier/3D/data/ABC}{\underline{Arnold-Beltrami-Childress (ABC) flow}}: \\ \\
 Classic spatially periodic ABC flow. We consider both the steady and unsteady version.
 \item \href{https://github.com/EncinasBartos/TBarrier/tree/main/TBarrier/3D/data/Turbulence}{\underline{Turbulence}}: \\ \\
 Three dimensional turbulent channel flow data from John Hopkins Research Center)
 \end{enumerate}
  \item \href{https://github.com/EncinasBartos/TBarrier/tree/main/TBarrier/3D/demos}{\underline{demos}}
\begin{enumerate}
 \item \href{https://github.com/EncinasBartos/TBarrier/tree/main/TBarrier/3D/demos/AdvectiveBarriers}{\underline{AdvectiveBarriers}} \\
  \begin{itemize}
  \item \href{https://github.com/EncinasBartos/TBarrier/tree/main/TBarrier/3D/demos/AdvectiveBarriers/FTLE}{\underline{FTLE}} (see $ \textit{Hyperbolic LCS from the finite-time Lyapunov exponent} $) \\ \\
FTLE-field applied to ABC data. \\ \\
\item \href{https://github.com/EncinasBartos/TBarrier/tree/main/TBarrier/3D/demos/AdvectiveBarriers/TSE}{\underline{TSE}} (see $ \textit{Quasi-objective, single-trajectory diagnostics for transport
barriers} $) \\ \\
TSE field applied to ABC and turbulent flow channel data.\\ \\
  \item \href{https://github.com/EncinasBartos/TBarrier/tree/main/TBarrier/3D/demos/AdvectiveBarriers/TRA}{\underline{TRA}} (see $ \textit{Quasi-objective, single-trajectory diagnostics for transport
barriers} $) \\ \\
TRA field applied to ABC and turbulent flow channel data.\\ \\
\item \href{https://github.com/EncinasBartos/TBarrier/tree/main/TBarrier/2D/demos/AdvectiveBarriers/LAVD}{\underline{LAVD}}: (see $ \textit{Elliptic LCSs from the Lagrangian-averaged vorticity deviation} $ and $ \textit{ LAVD for 3D flows} $) \\ \\
LAVD field applied to ABC data.\\ \\
\item \href{https://github.com/EncinasBartos/TBarrier/tree/main/TBarrier/3D/demos/AdvectiveBarriers/UnifiedLCSTheory}{\underline{UnifiedLCSTheory}}: (see $ \textit{Unified variational theory of elliptic and hyperbolic LCS in 3D} $) \\ \\
Extract LCS from the $ \xi_2 $ eigenvector field of the Cauchy-Green strain tensor from the ABC data.
\\ \\
\item \href{https://github.com/EncinasBartos/TBarrier/tree/main/TBarrier/3D/demos/AdvectiveBarriers/PoincareMap}{\underline{PoincareMap}}: \\ \\
Classical Poincare map applied to ABC data.
\\ \\
 \end{itemize}
 \item \href{https://github.com/EncinasBartos/TBarrier/tree/main/TBarrier/3D/demos/ActiveBarriers}{\underline{ActiveBarriers}} \\ \\
The active barriers are extracted from the three-dimensional channel flow data (see $ \textit{3D turbulent channel flow} $) and/or from the ABC data \newline (see $ \textit{Quasi-objective, single-trajectory diagnostics for transport barriers} $). \\ \\ Lagrangian end Eulerian active barriers to vorticity and linear momentum are extracted using the active FTLE, active PRA, active TSE and active TRA. \\
 \begin{itemize}
  \item \href{https://github.com/EncinasBartos/TBarrier/tree/main/TBarrier/3D/demos/ActiveBarriers/aFTLE}{\underline{aFTLE}} (see $ \textit{Active FTLE (aFTLE) and active TSE (aTSE)} $) \\
    \item \href{https://github.com/EncinasBartos/TBarrier/tree/main/TBarrier/3D/demos/ActiveBarriers/aTSE}{\underline{aTSE}} (see $ \textit{Active FTLE (aFTLE) and active TSE (aTSE)} $) \\
  \item \href{https://github.com/EncinasBartos/TBarrier/tree/main/TBarrier/3D/demos/ActiveBarriers/aPRA}{\underline{aPRA}} (see $ \textit{Active PRA (aPRA) and active TRA (aTRA)} $) \\
  \item \href{https://github.com/EncinasBartos/TBarrier/tree/main/TBarrier/3D/demos/ActiveBarriers/aTRA}{\underline{aTRA}} (see $ \textit{Active PRA (aPRA) and active TRA (aTRA)} $) \\
 \end{itemize}
\end{enumerate}
\item \href{https://github.com/EncinasBartos/TBarrier/tree/main/TBarrier/3D/subfunctions}{\underline{subfunctions}}: \\ \\
This folder contains frequently used functions to compute trajectories from two-dimensional velocity data, evaluate the gradient of the flowmap/velocity, the classic Cauchy-Green strain tensor, etc...
\end{enumerate}
\end{enumerate}


\end{document}